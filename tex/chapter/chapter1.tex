%%%%%%%%%%%%%%%%%%%%%%%%%%%%%%%%%%%%%%%%%%%%%%%%%%%%%
%%%%%%%%%%%%%%%%%%%%%%%%%%%%%%%%%%%%%%%%%%%%%%%%%%%%%
\chapter{稠密矩阵和数组处理}

%%%%%%%%%%%%%%%%%%%%%%%%%%%%%%%%%%%%%%%%%%%%%%%%%%%%%
\section{矩阵类}
原文地址:\href{http://eigen.tuxfamily.org/dox/group__TutorialMatrixClass.html}{MatrixClass}

%===================================================%
\subsection{Matrix的前三个模板参数}
Matrix类有6个模板参数,但是就现在而言掌握前三个参数已经足够了。剩下的三个参数,带有默认值,在后面的内容中会进行讨论。

Matrix前三个强制的模板参数是:

\verb|Matrix<typename Scalar, int RowsAtCompileTime, int ColsAtCompileTime>|

\begin{itemize}
\item Scalar, 是标量类型,如每个元素的类型。也就是说,如果你想要一个存储float的矩阵,那么scalar就是float。
\item RowsAtCompileTime和ColsAtCompileTime是在编译期间需要确定的矩阵的行数和列数。
\end{itemize}

我们提供了很多方便的typedef来覆盖大多数场景。比如,\verb|Matrix4f|是一个4x4的float矩阵,在eigen中是如下定义的:

\verb|typedef Matrix<float,4,4> Matrix4f;|

接下来会讨论一些通用的typedefs。

%===================================================%
\subsection{Vectors}

正如上面所说,Vector就是矩阵的一个特例。在Eigen,以Vector开头的都是列向量。

举个例子,Vector3f就是3个float元素的列向量,在Eigen中定义如下:

\verb|typedef Matrix<float,3,1> Vector3f;|

对于行向量也有类似的定义,如下:

\verb|typedef Matrix<int, 1, 2> RowVector2i;|

%===================================================%
\subsection{Dynamic特殊值}

当然,Eigen并没有限制编译的时候必须知道矩阵的大小。RowsAtCompileTime和ColsAtCompileTime模板参数可以设定为特殊的值,\verb|Dynamic|,用来表示编译期间大小是未知的。在Eigen中,这样的大小被叫做动态大小,在编译期间确定的大小被叫做固定大小。以MatrixXd为例,如下:

\verb|typedef Matrix<double, Dynamic, Dynamic> MatrixXd;|

类似的定义VectorXi,如下:

\verb|typedef Matrix<int, Dynamic, 1> VectorXi;|

当然也可以定义固定行数,不固定列数,如:\verb|Matrix<float, 3, Dynamic>|。

%===================================================%
\subsection{构造函数}
通常提供默认构造函数,默认构造函数不会进行内存分配,也不会进行矩阵元素的初始化。可以如下:

\begin{lstlisting}[style=Cpp]
Matrix3f a;
MatrixXf b;
\end{lstlisting}

这里:
\begin{itemize}
\item a,是一个3x3的矩阵,一个没有初始化的float[9]数组;
\item b不固定大小的矩阵,当前大小为0x0,元素数组还没有被分配空间。
\end{itemize}

还有带大小的构造函数。对于矩阵而言,第一个参数通常是行数。对于向量而言,只需要传入向量的大小。构造函数内部会根据大小分配对应的内存空间,但是并不会对元素进行初始化。

\begin{lstlisting}[style=Cpp]
MatrixXf a(10,15);
VectorXf b(30);
\end{lstlisting}

这里:
\begin{itemize}
\item a是一个10x15大小可以动态设定的矩阵,内存空间已经分配,但是还没有进行初始化。
\item b是一个大小为30的可以动态设定大小的向量,内存空间已经分配,但是还没有进行初始化。
\end{itemize}

为了使得固定大小和动态设定大小的矩阵的接口保持一致,对于固定大小的矩阵,构造的时候传入大小也是合法的。如:

\verb|Matrix3f a(3,3);|

最后,我们也提供了一些方便固定大小的vector初始化的构造,如下:

\begin{lstlisting}[style=Cpp]
Vector2d a(5.0, 6.0);
Vector3d b(5.0, 6.0, 7.0);
Vector4d c(5.0, 6.0, 7.0, 8.0);
\end{lstlisting}

%===================================================%
\subsection{元素访问}
Eigen中主要元素的访问和修改,是通过对括号进行了重载实现的。对于矩阵而言,第一个参数总是行的index。对于向量而言,只需要传递一个参数即可。index是从0开始计数的。例子如下:

\begin{lstlisting}[style=Cpp]
#include <iostream>
#include <Eigen/Dense>
using namespace Eigen;
int main()
{
	MatrixXd m(2,2);
	m(0,0) = 3;
	m(1,0) = 2.5;
	m(0,1) = -1;
	m(1,1) = m(1,0) + m(0,1);
	std::cout << "Here is the matrix m:\n" << m << std::endl;
	VectorXd v(2);
	v(0) = 4;
	v(1) = v(0) - 1;
	std::cout << "Here is the vector v:\n" << v << std::endl;
}
\end{lstlisting}
注意单个参数的方式(\verb|m(index)|)并不是一定针对向量,也可以用于矩阵,用于通过index来访问内部存储的一维数组中对应的元素。但是这个得到的结果依赖于矩阵的存储方式。Eigen默认采用的是以列为主的存储方式,不过也可以通过参数的设置改变成以行为主的存储方式。

\verb|[]|操作符也用于向量元素的访问,不过需要注意的是,c++不允许操作符\verb|[]|传入多于一个的参数。因此该操作符仅能用于向量的访问。

%===================================================%
\subsection{逗号初始化}
矩阵和向量能够很方便的通过逗号初始化的方式进行初始化,如下:
\begin{lstlisting}[style=Cpp]
Matrix3f m;
m << 1, 2, 3,
4, 5, 6,
7, 8, 9;
std::cout << m;

// output is:
1 2 3
4 5 6
7 8 9
\end{lstlisting}

\subsection{改变大小}

矩阵当前的大小可以通过\verb|rows(),cols(),size()|获取。这些方法返回矩阵的行数,列数,以及元素的个数。对于大小动态设定的矩阵的resize,可以通过函数\verb|resize()|实现。如下:
\begin{lstlisting}[style=Cpp]
#include <iostream>
#include <Eigen/Dense>
using namespace Eigen;
int main()
{
	MatrixXd m(2,5);
	m.resize(4,3);
	std::cout << "The matrix m is of size "
	<< m.rows() << "x" << m.cols() << std::endl;
	std::cout << "It has " << m.size() << " coefficients" << std::endl;
	VectorXd v(2);
	v.resize(5);
	std::cout << "The vector v is of size " << v.size() << std::endl;
	std::cout << "As a matrix, v is of size "
	<< v.rows() << "x" << v.cols() << std::endl;
}

// 结果:
The matrix m is of size 4x3
It has 12 coefficients
The vector v is of size 5
As a matrix, v is of size 5x1
\end{lstlisting}

如果矩阵实际的元素个数(size)没有发生变化的话,那么resize操作不会改变元素的数值,否则,该操作是破坏性的:元素的数值很有可能被更改。如果你想要在resize的时候不改变元素的数值,可以选用另一个方法,\verb|conservativeResize()|。

当然这些接口对于固定大小的矩阵也能够调用,但是,你并不能改变矩阵的大小。试图改变,会引起程序的中断,但是如果大小没有发生变化,那么代码是合理的。如:
\begin{lstlisting}[style=Cpp]
#include <iostream>
#include <Eigen/Dense>
using namespace Eigen;
int main()
{
	Matrix4d m;
	m.resize(4,4); // no operation
	std::cout << "The matrix m is of size "
	<< m.rows() << "x" << m.cols() << std::endl;
}

// output
The matrix m is of size 4x4
\end{lstlisting}

%===================================================%
\subsection{ 赋值和改变大小}
赋值操作是将一个矩阵拷贝到另一个矩阵的行为,使用=操作符。Eigen会自动改变左边的矩阵大小,和=右边的大小保持一致。
\begin{lstlisting}[style=Cpp]
MatrixXf a(2,2);
std::cout << "a is of size " << a.rows() << "x" << a.cols() << std::endl;
MatrixXf b(3,3);
a = b;
std::cout << "a is now of size " << a.rows() << "x" << a.cols() << std::endl;

// output
a is of size 2x2
a is now of size 3x3
\end{lstlisting}
当然,如果左边是固定大小的矩阵,那么改变大小是不被允许的。如果你不希望这种情况自动发生,那么你可以禁用它。(TODO)

%===================================================%
\subsection{固定大小和动态大小的比较}

什么时候要用固定大小的矩阵,如Matrix4f,以及什么时候更加倾向于动态矩阵,如MatrixXf。简单的答案是:针对非常小的矩阵使用固定大小的,对于大的矩阵使用动态大小的。对于小的矩阵,尤其是小于16(粗略的估计),使用固定大小的矩阵,对于性能更为有利,因为这样能够避免内存的分配以及释放等。在内部,固定大小的矩阵就是一个数组。如\verb|Matrix4f mymatrix;|的行为和\verb|float mymatrix[16];|的行为基本上是一致的。

作为比较,动态大小的矩阵,内存空间是分配再堆上面的,如\verb|MatrixXf mymatrix(rows,columns);|和\verb|float *mymatrix = new float[rows*columns];|比较类似的,不同的是前者存储了rows和columns作为成员变量。

使用固定大小的限制是,在编译阶段你必须要知道矩阵的大小。对于大的矩阵,使用固定大小带来的性能的提升可以被忽略不计。更糟糕的是,试图创建一个非常大的固定大小的矩阵,可能会导致栈溢出,因为Eigen将分配得到的数组作为局部变量,而局部变量存在于栈上。最后,和环境相关的是,在使用动态大小的时候,Eigen会尽可能的使用向量化(使用SIMD指令)。

%===================================================%
\subsection{可选的模板参数}
在这一页开头,介绍了Matrix有六个模板参数,但是到目前为止,我们仅仅讨论了3个,剩下的参数如下:
\begin{lstlisting}[style=Cpp]
Matrix<typename Scalar,
int RowsAtCompileTime,
int ColsAtCompileTime,
int Options=0,
int MaxRowsAtCompileTime = RowsAtCompileTime,
int MaxColsAtCompileTime = ColsAtCompileTime>
\end{lstlisting}
\begin{itemize}
\item Options:一个位字段,这里讨论一个呗,RowMajor。用来定义矩阵内部存储的是行为主的。默认情况下是列为主的。可以这么使用\verb|Matrix<float,3,3,RowMajor>| 。
\item MaxRowsAtCompileTime和MaxColsAtCompileTime,使用这两个参数的最大个一个原因是,你需要避免内存的动态分配,举个例子,下面的矩阵使用了12个float的数组,但没有动态分配内存空间。\verb|Matrix<float,Dynamic,Dynamic,0,3,4>|。
\end{itemize}

%===================================================%
\subsection{方便的typedefs}
Eigen中定义了如下的typedefs:
\begin{itemize}
\item MatrixNt的格式用于定义\verb|Matrix<type,N,N>|。如:MatrixXi为\verb|Matrix<int,Dynamic,Dynamic>|。
\item VectorNt的格式用于定义\verb|Matrix<type,N,1>|。如:Vector2f为\verb|Matrix<float,2,1>|。
\item RowVectorNt的格式用于定义\verb|Matrix<type,1,N>|。如:RowVector3d为\verb|Matrix<double,1,3>|。
\end{itemize}

上面的格式中,

\begin{itemize}
\item N可以是2,3,4,X中的任意一个。
\item t可以是i(int), f(float), d(double), cf(complex<float>), cd(complex<double>)中的任意一个。这里给出了5个标量类型,但不是说明,一定得这几个,可以进行扩展。
\end{itemize} 

%%%%%%%%%%%%%%%%%%%%%%%%%%%%%%%%%%%%%%%%%%%%%%%%%%%%%
\section{矩阵和向量运算}
这一节的主要内容为:
\begin{itemize}
\item 稠密矩阵和数组计算
\item 主要提供综述和在矩阵、向量和标量之间怎样使用的一些细节
\end{itemize}

%===================================================%
\subsection{介绍}
\verb|Eigen|通过重载常用的C++运算符,例如+,-,*或者通过特殊方法,例如点乘,叉乘等等。对于矩阵类(矩阵和向量)运算符重载仅仅支持线性代数运算。例如,matrix1 * matrix2为矩阵与矩阵的乘积,vector + scalar 向量与标量相加是不允许。如果想要使用各种数组运算,而不是线性代数运算。

\subsection{加法与减法}
当然,左手边与右手边的行和列必须相同,也必须有同样的标量类型,Eigen不能自动类型转换。这里有运算符是:
\begin{itemize}
\item 二元运算符 + ,a + b
\item 二元运算符 - ,a - b
\item 一元运算符 - ,-a
\item 复合运算符 += , a += b
\item 复合运算符 -= , a -= b
\end{itemize}
\begin{lstlisting}[style=Cpp]
#include <iostream>
#include <Eigen/Dense>

int main()
{
	Eigen::Matrix2d a;
	a << 1, 2,
	3, 4;
	Eigen::MatrixXd b(2,2);
	b << 2, 3,
	1, 4;
	std::cout << "a + b =\n" << a + b << std::endl;
	std::cout << "a - b =\n" << a - b << std::endl;
	std::cout << "Doing a += b;" << std::endl;
	a += b;
	std::cout << "Now a =\n" << a << std::endl;
	Eigen::Vector3d v(1,2,3);
	Eigen::Vector3d w(1,0,0);
	std::cout << "-v + w - v =\n" << -v + w - v << std::endl;
}
\end{lstlisting}

输出的结果为:
\begin{lstlisting}
a + b =
3 5
4 8
a - b =
-1 -1
2  0
Doing a += b;
Now a =
3 5
4 8
-v + w - v =
-1
-4
-6
\end{lstlisting}

\subsection{标量乘法与除法}
\begin{itemize}
\item 二元运算符 *, matrix * scalar
\item 二元运算符 *, scalar * matrix
\item 二元运算符 /,matrix / scalar
\item 复合运算符 *=,matrix *= scalar
\item 复合运算符 /=, matrix /= scalar
\end{itemize}

\begin{lstlisting}[style=Cpp]
#include <iostream>
#include <Eigen/Dense>

int main()
{
	Eigen::Matrix2d a;
	a << 1, 2,
	3, 4;
	Eigen::Vector3d v(1,2,3);
	std::cout << "a * 2.5 =\n" << a * 2.5 << std::endl;
	std::cout << "0.1 * v =\n" << 0.1 * v << std::endl;
	std::cout << "Doing v *= 2;" << std::endl;
	v *= 2;
	std::cout << "Now v =\n" << v << std::endl;
}
\end{lstlisting}

输出的结果为:
\begin{lstlisting}
a * 2.5 =
2.5   5
7.5  10
0.1 * v =
0.1
0.2
0.3
Doing v *= 2;
Now v =
2
4
6
\end{lstlisting}

\subsection{关于表达式模板的说明}
这是在本章中描述的一个高级主题,但是这里只是提到它的使用,在Eigen中,运算重载符,例如Operate+不会使用任何自身的计算,仅仅返回一个表达式对象来描述使用的计算。实际计算是在之后,当整个表达式被求值时,通常在Operator=中。虽然这听起来很沉重,但是任何现代编译器都能够优化这种抽象运算符,从而得到完全优化的代码的结果。例如:

\begin{lstlisting}[style=Cpp]
VectorXf a(50), b(50), c(50), d(50);
...
a = 3*b + 4*c + 5*d;
\end{lstlisting}

Eigen 编译器仅仅只做一次循环,以至于数组只遍历一次,简化(忽略SIMD优化器),循环如下:
\begin{lstlisting}[style=Cpp]
for(int i = 0; i < 50; ++i)
a[i] = 3*b[i] + 4*c[i] + 5*d[i];
\end{lstlisting}
因此,不应该担心使用带有特征的相对较大的算术表达式:这只会为Eigen提供更多的优化。

\subsection{转置和共轭}

对矩阵的转置、共轭和共轭转置分别由成员函数transpose(),conjugate(),adjoint()实现。
\begin{lstlisting}[style=Cpp]
MatrixXcf a = MatrixXcf::Random(2,2);
cout << "Here is the matrix a\n" << a << endl;

cout << "Here is the matrix a^T\n" << a.transpose() << endl;

cout << "Here is the conjugate of a\n" << a.conjugate() << endl;

cout << "Here is the matrix a^*\n" << a.adjoint() << endl;
\end{lstlisting}
输出的结果为:
\begin{lstlisting}
Here is the matrix a
(-0.211,0.68) (-0.605,0.823)
(0.597,0.566)  (0.536,-0.33)
Here is the matrix a^T
(-0.211,0.68)  (0.597,0.566)
(-0.605,0.823)  (0.536,-0.33)
Here is the conjugate of a
(-0.211,-0.68) (-0.605,-0.823)
(0.597,-0.566)    (0.536,0.33)
Here is the matrix a^*
(-0.211,-0.68)  (0.597,-0.566)
(-0.605,-0.823)    (0.536,0.33)
\end{lstlisting}
对于实数矩阵,共轭不会发生操作,所以adjoint()=transpose()。

对于基本的重载运算符、transpose()和adjoint()简单的返回一个代理对象,而不是做实际的转置,如果你做b = a.transpose(),然后转置在结果被写进b的同时被求值,然而这里是一个复杂的问题,如果你做 a = a.transpose(),那么在转置的计算完成之前,Eigen开始把结果写进a中,因此,并不是期待那样,公式a = a.transpose()用它替换a

\begin{lstlisting}[style=Cpp]
Matrix2i a; a << 1, 2, 3, 4;
cout << "Here is the matrix a:\n" << a << endl;

a = a.transpose(); // !!! do NOT do this !!!
cout << "and the result of the aliasing effect:\n" << a << endl;
\end{lstlisting}
输出的结果为:
\begin{lstlisting}
Here is the matrix a:
1 2
3 4
and the result of the aliasing effect:
1 2
2 4  (0.536,0.33)
\end{lstlisting}

这里所谓的别名问题,在调试模型,在没有断言时,将自动检测此类常见的缺陷。
对于In-place转置,对于例子a = a.transpose(),简单地使用transposeInPlace()函数。
\begin{lstlisting}[style=Cpp]
MatrixXf a(2,3); a << 1, 2, 3, 4, 5, 6;
cout << "Here is the initial matrix a:\n" << a << endl;

a.transposeInPlace();
cout << "and after being transposed:\n" << a << endl;
\end{lstlisting}
输出的结果为:
\begin{lstlisting}
Here is the initial matrix a:
1 2 3
4 5 6
and after being transposed:
1 4
2 5
3 6
\end{lstlisting}
对于复杂的矩阵这里还有adjointInPlace() 函数。

\subsection{ 矩阵相乘和矩阵与向量相乘}

矩阵相乘使用operator* ,由于向量也是一种特殊的矩阵,这里是一种隐式处理,所以矩阵与向量乘积是一种特殊矩阵相乘,也是向量与向量的外积,因此,所有的这些运算操作同两个重载符:

\begin{itemize}
\item 二元重载符*,a * b
\item 三元重载符 *= ,a *= b(),在右边相乘:a*=b等于a = a*b。
\end{itemize}
\begin{lstlisting}[style=Cpp]
#include <iostream>
#include <Eigen/Dense>

int main()
{
	Eigen::Matrix2d mat;
	mat << 1, 2,
	3, 4;
	Eigen::Vector2d u(-1,1), v(2,0);
	std::cout << "Here is mat*mat:\n" << mat*mat << std::endl;
	std::cout << "Here is mat*u:\n" << mat*u << std::endl;
	std::cout << "Here is u^T*mat:\n" << u.transpose()*mat << std::endl;
	std::cout << "Here is u^T*v:\n" << u.transpose()*v << std::endl;
	std::cout << "Here is u*v^T:\n" << u*v.transpose() << std::endl;
	std::cout << "Let's multiply mat by itself" << std::endl;
	mat = mat*mat;
	std::cout << "Now mat is mat:\n" << mat << std::endl;
}
\end{lstlisting}
输出的结果为:
\begin{lstlisting}
Here is mat*mat:
7 10
15 22
Here is mat*u:
1
1
Here is u^T*mat:
2 2
Here is u^T*v:
-2
Here is u*v^T:
-2 -0
2  0
Let's multiply mat by itself
Now mat is mat:
7 10
15 22
\end{lstlisting}

提示:如果你阅读上面关于表达式模板的段落和可能导致别名问题m=m*m的担忧,Eigen把矩阵相乘作为特殊情况,引用一个临时变量,因此将它编译为m=m*m

\verb|tmp = m*m;|

\verb|m = tmp;|

如果知道矩阵相乘能够安全的计算成目标矩阵而没有别名问题,可以使用noalias()函数来避免临时变量。

\verb|c.noalias() += a * b;|

对于更多的细节,见\href{https://eigen.tuxfamily.org/dox/group__TopicAliasing.html}{Aliasing}。提示:对于BLAS使用者担心的性能问题,表达式\verb|c.noalias() -= 2 * a.adjoint() * b|全部优化和触发一个gemm-like函数调用。

\subsection{点乘和叉乘}

对于点乘与叉乘,使用dot()和cross()方法,当然点乘也可以用u.adjoint()*v表示为1x1矩阵。
\begin{lstlisting}[style=Cpp]
#include <iostream>
#include <Eigen/Dense>

int main()
{
	Eigen::Vector3d v(1,2,3);
	Eigen::Vector3d w(0,1,2);
	
	std::cout << "Dot product: " << v.dot(w) << std::endl;
	double dp = v.adjoint()*w; // automatic conversion of the inner product to a scalar
	std::cout << "Dot product via a matrix product: " << dp << std::endl;
	std::cout << "Cross product:\n" << v.cross(w) << std::endl;
}
\end{lstlisting}
输出的结果为:
\begin{lstlisting}
Dot product: 8
Dot product via a matrix product: 8
Cross product:
1
-2
1
\end{lstlisting}
记住叉乘向量的维度为3,向量的点乘任意维度,当使用复数时,Eigen点乘在第一个变量中是共轭线性的,在第二个变量中也是线性的。

\subsection{基本算术规约操作}

Eigen提供了一些对于矩阵或向量的规约操作,如sum(),prod(),maxCoeff()和minCoeff()

\begin{lstlisting}[style=Cpp]
#include <iostream>
#include <Eigen/Dense>

using namespace std;
int main()
{
	Eigen::Matrix2d mat;
	mat << 1, 2,
	3, 4;
	cout << "Here is mat.sum():       " << mat.sum()       << endl;
	cout << "Here is mat.prod():      " << mat.prod()      << endl;
	cout << "Here is mat.mean():      " << mat.mean()      << endl;
	cout << "Here is mat.minCoeff():  " << mat.minCoeff()  << endl;
	cout << "Here is mat.maxCoeff():  " << mat.maxCoeff()  << endl;
	cout << "Here is mat.trace():     " << mat.trace()     << endl;
}
\end{lstlisting}
输出的结果为:
\begin{lstlisting}
Here is mat.sum():       10
Here is mat.prod():      24
Here is mat.mean():      2.5
Here is mat.minCoeff():  1
Here is mat.maxCoeff():  4
Here is mat.trace():     5
\end{lstlisting}

矩阵的迹,由函数trace()返回,迹是矩阵对角系数之和,也可以使用a.diagonal().sum()高效计算。
这里还存在minCoeff 和maxCoeff 函数返回各系数的坐标。

\begin{lstlisting}[style=Cpp]
Matrix3f m = Matrix3f::Random();
std::ptrdiff_t i, j;
float minOfM = m.minCoeff(&i,&j);
cout << "Here is the matrix m:\n" << m << endl;
cout << "Its minimum coefficient (" << minOfM 
<< ") is at position (" << i << "," << j << ")\n\n";

RowVector4i v = RowVector4i::Random();
int maxOfV = v.maxCoeff(&i);
cout << "Here is the vector v: " << v << endl;
cout << "Its maximum coefficient (" << maxOfV 
<< ") is at position " << i << endl;
\end{lstlisting}
输出的结果为:
\begin{lstlisting}
Here is the matrix m:
0.68  0.597  -0.33
-0.211  0.823  0.536
0.566 -0.605 -0.444
Its minimum coefficient (-0.605) is at position (2,1)

Here is the vector v:  1  0  3 -3
Its maximum coefficient (3) is at position 2
\end{lstlisting}

\subsection{操作的有效性}

Eigen检查执行的操作的有效性。如果可能,它会在编译时检查它们,从而产生编译错误。这些错误消息可能又长有难看,但是Eigen用大写字母 \textbf{UPPERCASE\_LETTERS\_SO\_IT\_STANDS\_OUT}写重要的消息。例如:
\begin{lstlisting}[style=Cpp]
Matrix3f m;
Vector4f v;
v = m*v; // Compile-time error: YOU_MIXED_MATRICES_OF_DIFFERENT_SIZES
\end{lstlisting}

当然,在许多情况下,例如在检查动态大小时,无法在编译时执行检查。然后使用运行时断言。这意味着如果在“调试模式”下运行,程序在执行非法操作时将使用错误消息中止,如果关闭断言,程序可能会崩溃。

\begin{lstlisting}[style=Cpp]
MatrixXf m(3,3);
VectorXf v(4);
v = m * v; // Run-time assertion failure here: "invalid matrix product"
\end{lstlisting}

%%%%%%%%%%%%%%%%%%%%%%%%%%%%%%%%%%%%%%%%%%%%%%%%%%%%%
\section{数组类和coefficient-wise操作}

